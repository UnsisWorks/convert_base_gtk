\documentclass[letterpaper,12pt]{extarticle}%Preambulo
\usepackage[utf8]{inputenc}%Preambulo
\usepackage[spanish,mexico]{babel}%Preambulo
\usepackage{ae}
\usepackage{amsmath,amssymb,amsfonts,latexsym,cancel}%Preambulo
\usepackage{hyperref}%Preambulo
\usepackage[pdftex]{graphicx}%Preambulo
\usepackage{wrapfig}%Preambulo
\usepackage[rflt]{floatflt}%Preambulo
\usepackage{fancyhdr}%%Pre?mbulo
\usepackage{mathptmx}%%Pre?mbulo
\usepackage{float}%%Pre?mbulo
\usepackage{longtable,multirow,booktabs}%%Pre?mbulo
%\usepackage{cite}
\usepackage{wrapfig}%%Pre?mbulo
\usepackage[rflt]{floatflt}%%Pre?mbulo
\usepackage{natbib} %%Pre?mbulo
\usepackage{multicol}%%Pre?mbulo
\usepackage{caption}%%Pre?mbulo
\usepackage{geometry}
\usepackage{wrapfig}
\usepackage{adjustbox}
\usepackage{amsmath}
\usepackage{parskip}
\usepackage{tikz}
\usepackage{lipsum}
\usepackage{xcolor}
\usepackage[T1]{fontenc}

\captionsetup{
       font=small,
       labelfont=bf,
       tableposition=top,
       hypcap=false
    }

\DeclareGraphicsExtensions{.pdf, .png, .jpg, .PNG, .JPG}%Cuando ponemos im?genes ya no es necesario poner la extensi?n

%%%FORMATO DE LA P?GINA%%%
\textheight = 21cm %Medidas de la  p?gina
\textwidth = 18cm  %Medidas de la p?gina
\topmargin = -1cm  %Medidas de la p?gina    
\oddsidemargin = -1cm %Medidas de la p?gina
\pagestyle{fancy} %Dise?o de la p?gina
\lhead{Universidad de la Sierra Sur}%%LeftHead
\chead{\includegraphics[width=1cm, height=1cm]{imag//logL}}%%CenterHead
%\lfoot{USM}
\rhead{Licenciatura en Informática}%%RightHead
%\lfoot{Gonz?lez Chico Juan Daniel} %Pie de pagina izquierdo

\setlength{\columnsep}{7mm}%Comandos para el formato de la p?gina
%\setlength{\parindent}{4em}%Sangr?a al comenzar un nuevo p?rrafo
\setlength{\parindent}{0.5in}
%\setlength{\parindent}{4em}%Sangr?a al comenzar un nuevo p?rrafo
\setlength{\parskip}{1em}%distancia entre p?rrafos
\renewcommand{\baselinestretch}{1.0}% Espacio entre l?nea y l?nea
\setlength{\headheight}{33pt}

\begin{document}

    \begin{titlepage}
		% Mine page for add image at scale with footer image in 
		\begin{figure}[ht]
		   \minipage{0.76\textwidth}
				\includegraphics[width=4cm]{imag//logColor.jpg}
				\label{escudoFI}
		   \endminipage
		   \minipage{0.32\textwidth}

				\includegraphics[height = 4.5cm ,width=4cm]{imag//logBN.jpg} 
				\label{EscuoUNAM}
			\endminipage
				%%\vspace{-1cm}
		\end{figure}
		
		\vspace{0.5cm}
		
		\begin{center}
			\LARGE UNIVERSIDAD DE LA SIERRA SUR \\
			\vspace{0.3cm}
			\LARGE Instituto de Informática
			
			\vspace{.7cm} {\LARGE  \textbf{Programa de conversión de bases} \\}

			% Incrementamos el interlineado:
			\vspace{.7cm} {\LARGE Labortorio de Sistemas Digitales}

			% Restauramos el interlineado:
			\vspace{.5cm}
			\begin{center}

				\LARGE{ \textbf{Alumnos:}}\\%% \textbf son negritas
        \LARGE{Elietzer Jared, kevin Emmanuel}\\%% \it es letra it?lica
				\vspace{0.5cm}
				\textbf{Profesor:}  Dr. Alejandro Jarillo Silva \\
				\vspace{0.5cm}
				\textbf{Grupo:}  306
				
			\end{center}
			
			\vspace{1cm} \today
		\end{center}
	\end{titlepage}

    \newpage
    \tableofcontents
    \newpage
    
    \begin{center}
    \textbf{ Alumno:}\\[3mm]
    {\it Elietzer Jared}\\[3mm]
    {\it Kevin Emmanuel}\\[3mm]
    {\it Grupo 306}\\[3mm]
    {\it Convertidor de bases}\\[3mm]
    \end{center}
    

		% Inicio del documento

	% Section for links in doct e index
    \section{Introducción}
    
		Se desarrollara un programa de conversión 
		bases, el cual admitirá decimal, octal, binario, Hexadecimal
		y formato BCD. Para poder desplegarlo con formato
		se utilizará la tecnologia GTK+, biblioteca
		de el lenguaje de programación C 
		% % Coloca imagen y la centra en el espacio asignado
		% \begin{figure}[H]
		% \begin{center}
		% \includegraphics[width=4cm]{imag//logBN.jpg}
		% \caption{foto de escudo}
		% \label{figuraBN}
		% \end{center}
		% %%\vspace{-1cm}
 		% \end{figure}

		% ver la figura \ref*{figuraBN}
        
    \section{Objetivos}
			
    \begin{enumerate}
		\item Realizar una calculadora capaz de cambiar la base de un
		dado entre las bases: Decimal, Octal, Hexadecim,
		y Formato BCD

		\item Otorgar un diseño simple y util a la interfaz grafica
		para la facilitar el manejo del programa 
	
	\end{enumerate} 
		% Coloca imagen y la centra en el espacio asignado
		% \begin{figure}[H]
		% \begin{center}
		% \includegraphics[width=4cm]{imag//logColor.jpg}
		% \caption{foto de escudo}
		% Convertidor
		% \label{figura}
		% \end{center}
		% %%\vspace{-1cm}
 		% \end{figure}

		% ver figura \ref{figura} se observa....
    
		
	\section{Desarrollo}
	\subsection{Planteamiento del problema}
		Desarrollar un programa capaz de convertir un número de base decimal,
		Octal, Binaria, Hexadecimal o formato BCD a el resto de las bases, esto
		a ravés de una interfaz	gráfica. La implementación se realizará a través
		de C y su biblioteca gráfica GTK+ en su versión 3.24.20.

	\subsection{Diseño y creación de prototipos}
		Al inicio se pensaba darle un diseño de calculadora a el programa, pero 
		debido que tendrá bases específicas se opto por un diseño en el cual el 
		mismo programa determinara la base de entrada, para posteriormente calcular
		las demás bases.

		El programa tendráun diseño simple, que permita a el usuario ingresar
		de cualquier base admitida para obtenerlo en todas las demas bases 
		disponibles. Para lograr dicha flexibilidad se opto por el diseño
		mostrado en la figura \ref{ProgramDesign}

		% Coloca imagen de diseño de imagen
		\begin{figure}[H]
		\begin{center}
		\includegraphics[width=7cm]{imag//ProgramDesign.png}
		\caption{Diseño del programa}
		% Convertidor
		\label{ProgramDesign}
		\end{center}
		%%\vspace{-1cm}
 		\end{figure}

		Para limpiar los campos de entrada de texto se crearon dos formas, una mediante
		un botón, y otra a través de un click sobre cualquiera de los campos. El cambio de 
		metodo se da a través de el check con la leyenda: Limpiar con un click.
		
		Para el proceso de conversión se existen muchas formas de realizarlo, sin embargo
		se observo que pasar de Decimal a Binario, Octal y Hexadecimal se debe dividir el número
		decimal entre la base solicitada e ir concatenando los modulos de cada divición
		hasta obtener cero, como se muestra en la figura \ref{decToOct}, dondé se convierte de decimal
		a octal.

		% Coloca imagen de diseño de imagen
		\begin{figure}[H]
		\begin{center}
		\includegraphics[width=7cm]{imag//decimalToOctal.png}
		\caption{Decimal a Octal}
		% Convertidor
		\label{decToOct}
		\end{center}
		%\vspace{-1cm}
		\end{figure}

		Con dicha forma de convertir las bases desde un número decimal se llego a un algoritno el cual
		convierte desde decimal a octal, binario y Hexadecimal además de mostrar el resultado a el usuario
		y de forma opcional regresarlo mediante un return en caso de ser necesario, el cual se muestra en la 
		figura \ref{algDecToRest}.

		% Coloca imagen de diseño de imagen
		\begin{figure}[H]
			\begin{center}
			\includegraphics[width=10cm]{imag//algDecToRest.png}
			\caption{Algoritmo de conversión de bases}
			% Convertidor
			\label{algDecToRest}
			\end{center}
			%\vspace{-1cm}
			\end{figure}
	\subsection{Materiales}	
	
	\subsection{Procedimiento}
	\renewcommand{\labelenumi}{\arabic{enumi}.}
    
    	
    \section{Resultados}
    Se dscriben los resultados, puede ser a través de una tabla de mediciones
	\subsection{Mediciones}

	\begin{multicols}{2}
	    
	    \section{Conclusiones}

Se menciona las conclusiones de la práctica	    
	    	     %   \nocite{Cengel}
%	        \bibliographystyle{apalike}?
%            \bibliography{Libros.bib}
    \end{multicols}
    
	\newpage
	
	\section{Anexo de ecuaciones}
	
	En caso des ser necesario se deben agregar ecuaciones que se hayan empleado

	%Agrega todo lo ingresado tal cual
	\begin{verbatim}
		!"##$$%&%&/()
	\end{verbatim}

	\cfoot{\LaTeX}
\end{document}