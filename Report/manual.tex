\documentclass[letterpaper,12pt]{extarticle}%Preambulo
\usepackage[utf8]{inputenc}%Preambulo
\usepackage[spanish,mexico]{babel}%Preambulo
\usepackage{ae}
\usepackage{amsmath,amssymb,amsfonts,latexsym,cancel}%Preambulo
\usepackage{hyperref}%Preambulo
\usepackage[pdftex]{graphicx}%Preambulo
\usepackage{wrapfig}%Preambulo
\usepackage[rflt]{floatflt}%Preambulo
\usepackage{fancyhdr}%%Pre?mbulo
\usepackage{mathptmx}%%Pre?mbulo
\usepackage{float}%%Pre?mbulo
\usepackage{longtable,multirow,booktabs}%%Pre?mbulo
%\usepackage{cite}
\usepackage{wrapfig}%%Pre?mbulo
\usepackage[rflt]{floatflt}%%Pre?mbulo
\usepackage{natbib} %%Pre?mbulo
\usepackage{multicol}%%Pre?mbulo
\usepackage{caption}%%Pre?mbulo
\usepackage{geometry}
\usepackage{wrapfig}
\usepackage{adjustbox}
\usepackage{amsmath}
\usepackage{parskip}
\usepackage{tikz}
\usepackage{lipsum}
\usepackage{xcolor}
\usepackage[T1]{fontenc}

\captionsetup{
       font=small,
       labelfont=bf,
       tableposition=top,
       hypcap=false
    }

\DeclareGraphicsExtensions{.pdf, .png, .jpg, .PNG, .JPG}%Cuando ponemos im?genes ya no es necesario poner la extensi?n

%%%FORMATO DE LA P?GINA%%%
\textheight = 21cm %Medidas de la  p?gina
\textwidth = 18cm  %Medidas de la p?gina
\topmargin = -1cm  %Medidas de la p?gina    
\oddsidemargin = -1cm %Medidas de la p?gina
\pagestyle{fancy} %Dise?o de la p?gina
\lhead{Universidad de la Sierra Sur}%%LeftHead
\chead{\includegraphics[width=1cm, height=1cm]{imag//logL}}%%CenterHead
%\lfoot{USM}
\rhead{Licenciatura en Informática}%%RightHead
%\lfoot{Gonz?lez Chico Juan Daniel} %Pie de pagina izquierdo

\setlength{\columnsep}{7mm}%Comandos para el formato de la p?gina
%\setlength{\parindent}{4em}%Sangr?a al comenzar un nuevo p?rrafo
\setlength{\parindent}{0.5in}
%\setlength{\parindent}{4em}%Sangr?a al comenzar un nuevo p?rrafo
\setlength{\parskip}{1em}%distancia entre p?rrafos
\renewcommand{\baselinestretch}{1.0}% Espacio entre l?nea y l?nea
\setlength{\headheight}{33pt}

\begin{document}

    \begin{titlepage}
		% Mine page for add image at scale with footer image in 
		\begin{figure}[ht]
		   \minipage{0.76\textwidth}
				\includegraphics[width=4cm]{imag//logColor.jpg}
				\label{escudoFI}
		   \endminipage
		   \minipage{0.32\textwidth}

				\includegraphics[height = 4.5cm ,width=4cm]{imag//logBN.jpg} 
				\label{EscuoUNAM}
			\endminipage
				%%\vspace{-1cm}
		\end{figure}
		
		\vspace{0.5cm}
		
		\begin{center}
			\LARGE UNIVERSIDAD DE LA SIERRA SUR \\
			\vspace{0.3cm}
			\LARGE Instituto de Informática
			
			\vspace{.7cm} {\LARGE  \textbf{Programa de conversión de bases} \\}

			% Incrementamos el interlineado:
			\vspace{.7cm} {\LARGE Labortorio de Sistemas Digitales}

			% Restauramos el interlineado:
			\vspace{.5cm}
			\begin{center}

				\LARGE{ \textbf{Alumnos:}}\\%% \textbf son negritas
        \LARGE{Elietzer Jared, kevin Emmanuel}\\%% \it es letra it?lica
				\vspace{0.5cm}
				\textbf{Profesor:}  Dr. Alejandro Jarillo Silva \\
				\vspace{0.5cm}
				\textbf{Grupo:}  306
				
			\end{center}
			
			\vspace{1cm} \today
		\end{center}
\end{titlepage}

    \newpage
    \tableofcontents
    \newpage
    
    \begin{center}
    \textbf{ Alumno:}\\[3mm]
    {\it Elietzer Jared}\\[3mm]
    {\it Kevin Emmanuel}\\[3mm]
    {\it Grupo 306}\\[3mm]
    {\it Convertidor de bases}\\[3mm]
    \end{center}
    
    \begin{multicols}{2}

		% Inicio del documento

	% Section for links in doct e index
    \section{Introducción}
    
    Dar un pequeña introducción del tema central de la práctica
		% Coloca imagen y la centra en el espacio asignado
		\begin{figure}[H]
		\begin{center}
		\includegraphics[width=4cm]{imag//logBN.jpg}
		\caption{foto de escudo}
		\label{figuraBN}
		\end{center}
		%%\vspace{-1cm}
 		\end{figure}

		ver la figura \ref*{figuraBN}
        
    \section{Objetivos}

	\subsection{sub secion}

	Serán presentados en clase    
		% Coloca imagen y la centra en el espacio asignado
		\begin{figure}[H]
		\begin{center}
		\includegraphics[width=4cm]{imag//logColor.jpg}
		\caption{foto de escudo}
		Convertidor
		\label{figura}
		\end{center}
		%%\vspace{-1cm}
 		\end{figure}

		ver figura \ref{figura} se observa....

    
    \begin{enumerate}
    \item 
    \item
    \end{enumerate}
		
	\section{Desarrollo}
	
	Descripción general de la práctica
	
	\subsection{Materiales}	
	
	\begin{enumerate}
	    \item 
	    \item 
	    \item 
	    \item 
	    \item 
	    \item 
	    \item 
	    \item 
	    \item 
	\end{enumerate}
	
	\subsection{Procedimiento}
	\renewcommand{\labelenumi}{\arabic{enumi}.}
	\begin{enumerate}
		\item 
		
		\item 
		
		\item
		
		\item
		
		\item
		
	\end{enumerate}
    \end{multicols}
    
    	
    \section{Resultados}
    Se dscriben los resultados, puede ser a través de una tabla de mediciones
	\subsection{Mediciones}

	\begin{multicols}{2}
	    
	    \section{Conclusiones}

Se menciona las conclusiones de la práctica	    
	    	     %   \nocite{Cengel}
%	        \bibliographystyle{apalike}?
%            \bibliography{Libros.bib}
    \end{multicols}
    
	\newpage
	
	\section{Anexo de ecuaciones}
	
	En caso des ser necesario se deben agregar ecuaciones que se hayan empleado

	%Agrega todo lo ingresado tal cual
	\begin{verbatim}
		!"##$$%&%&/()
	\end{verbatim}

	\cfoot{\LaTeX}
\end{document}